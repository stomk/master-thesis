Due to the long-standing difficulty in assembling repetitive DNA, in-depth analyses of centromeric sequences have been limited \cite{Plohl2014}. Based on the high-quality genomes of the three medaka inbred strains which were recently assembled with PacBio long reads, this study has conducted a comprehensive analysis of the centromeric sequences in medaka. This is, to the best of the author's knowledge, the first in-depth study of centromeric sequence organization in fish species.

This study has revealed that the medaka centromeres are comprised of tandemly-repeated $\sim$150-bp satellite DNA which occupies 1--2\% of the genome. The satellite sequences are largely conserved between the three strains which are estimated to have separated 18 (Hd-rR and HNI) and 25 (HSOK and the previous two) million years ago (MYA) \cite{Setiamarga2009}. This conservation is consistent with the previous comparative study of 282 animal and plant species that observed the centromeric satellites are conserved among species within 50 million years after separation \cite{Melters2013}, whereas it is also known that in some closely-related species centromeric repeats are hardly conserved as a result of rapid diversification\cite{Lee2005}. On the other hand, the sequence organization including linear ordering of satellites were not conserved among the strains, suggesting that the centromeric arrays have undergone independent rearrangements after their separation.

The centromeric satellites are also conserved between some chromosomes. The hierarchical clustering of the chromosome-representative satellite monomers has revealed the presence of four subfamilies of the centromeric satellites in the medaka genome. Combining the results of the three strains, 22 out of the 24 chromosomes were classified to one or two subfamilies and the satellites of each subfamily are possessed by generally different subset of chromosomes. It should be noted that the classification could be imperfect because many chromosomes have relatively short repeat arrays in the assemblies which locate at the ends of much larger original arrays and the peripheral regions tend to accumulate mutations compared to highly-homogenenized core regions \cite{Smith1976, Schueler2001}. The presence of centromeric satellite subfamilies have been also observed in human, in which, similarly to the medaka satellite subfamilies, each subfamily belongs to generally different subsets of chromosomes \cite{Alexandrov2001}. It was also observed that the subfamilies exhibit different preference for the centromeric positions in chromosomes. For example, the medaka SF\,2 showed clear preference for acrocentric chromosomes. This is analogous to the human acrocentric chromosomes (chr.\,13, 14, 21 and 22) which share highly identical alpha-satellites \cite{Willard1991}. The human acrocentric chromosomes possess nucleolus organizers region (NOR) at their short arms and localize in close proximity in the nucleolus, where frequent sequence exchanges are believed to occur \cite{Willard1991}. However, the same process should never occur in the acrocentric chromosomes of medaka because it has been known that medaka has only single chromosome pair that has NOR \cite{Uwa1990}. It has been known that the teleost lineage underwent a whole-genome duplication (WGD) 336--404 MYA and since that medaka has largely retained the genomic structure without major chromosome rearrangements \cite{Kasahara2007}. However, the chromosome pairs that derived from the WGD did not necessarily possess similar centromeric satellite sequences, consistent with the Melters \textit{et al.}'s observation of centromeric satellite conservation within 50 MY.

It has been known that the centromeric sequence undergoes ``concerted evolution'' where copies of repetitive DNA or multi-copy gene families are more conserved within a species than between the species \cite{Dover1982, Willard1991, Charlesworth1994}. The concerted evolution is mediated by several molecular mechanisms such as unequal crossover, gene conversion and transpositions \cite{Dover1982}. The centromeric sequence conservation between the chromosomes irrelevant of the WGD event in medaka suggests that the medaka centromeres have followed the concerted evolution in the evolutionary process. The intermingled structure of the satellite arrays of different subfamilies and frequent orientation switch observed in some chromosomes implies the occurrence of sequence exchange between non-homologous chromosomes by gene conversion and subsequent amplification within the chromosome by unequal crossover between sister-chromatids and/or homologous chromosomes. In addition, the distinct preference of the centromeric positions between the satellite subfamilies might suggest that the sequence exchange by gene conversion may occur more frequently between the chromosomes with similar centromeric positions.

The medaka genome assemblies using PacBio long reads captured centromeric arrays in many chromosomes, including some long arrays over 100 kb. However the amount of centromeric repeats identified in the contigs that were anchored to the chromosomes was substantially below the estimated genomic abundance of the centromeric repeats in all the strains. Indeed large amount of the repeats were found in unanchored contigs. BAC/fosmid-end reads and Hi-C contact frequency data were used for the Hd-rR genome assembly, which successfully anchored a number of contigs containing centromeric repeats and resulted in higher portion of centromeric repeats to be anchored despite shorter read length than HSOK. This result emphasizes the complementary power of other long-range information methods in addition to the long read sequencing. The sum of the centromeric repeats in anchored and unanchored contigs did not reach the estimated genomic abundance, suggesting uncertain amount of the repeats were lost in the assembly process. Highly-homogenized HORs which might exist in the inner centromeric regions may have caused the under-represtation of centromeric repeats in the assembled contigs.

The centromeric sequence organization of three medaka strains revealed in this study provides insights into centromeric sequence evolution. Also, the characterized centromeric sequences provides a important basis for understanding the possible contribution of DNA sequences to the centromere specification and function.
