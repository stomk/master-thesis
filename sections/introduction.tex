\subsection*{The centromere}
  The centromere is a genomic region where kinetochore forms in mitosis and meiosis and plays a critical role for proper chromosome segregation.  The centromere is characterized by the presence of centromere-specific histone H3 variant CENH3 (also known as CENP-A), and underlying DNA is commonly comprised of repetitive sequences \cite{Plohl2014}. Despite its fundamental biological importance, the mechanisms of centromere specification and formation are still insufficiently understood \cite{McKinley2016}.


\subsection*{DNA sequences at the centromere}
  Repeat-based centromere is the most common structure in eukaryote species \cite{Plohl2014}. One of the major components of the repeat-based centromeres is a tandemly-repeated satellite DNA. Despite the conserved biological function of the centromere, DNA sequences at the centromere evolves rapidly (known as ``centromere paradox'') and the satellite sequences are generally species-specific \cite{Henikoff2001}. A well-known example is a 171-bp AT-rich alpha-satellite observed in human and many other primate species \cite{Willard1991}. In the human genome, tandemly-repeated alpha-satellites comprise hundred kilobase- to megabase-sized arrays in each chromosome \cite{Willard1987}. The satellite DNA can be highly diverged within a species, whereas the inner centromeric regions tend to have highly homogenous repeat arrays. In many primate species including human, the core centromeres consist of higher-order repeats (HORs) where several alpha-satellite monomers comprise a repeat unit which itself iterates tandemly with extremely high identity ($>$95\%) \cite{Willard1987}. Another major component of the repeat-based centromeres is retrotransposons, which has been widely observed in plant species \cite{Wang2009, Lermontova2015}. Satellites and retrotransposons are not mutually exclusive, rather intermingled structure of them are commonly observed \cite{Plohl2014}. These regional centromeres are flanked by heterochromatic pericentromeres. Centromeres and pericentromeres are characterized by distinct chromatin structures which are regulated by different epigenetic marks \cite{Verdaasdonk2011, McKinley2016}.

  While repeat-based regional centromeres are the most common structure in eukaryotes, some species have different types of centromeres. Budding yeast \textit{Saccharomyces cerevisiae} has a ``point centromere'' where $\sim$125-bp specific sequences form centromeres \cite{Hegemann1993}. In nematode \textit{C. elegans} and some species in insects and plants, the spindle microtubules attach all along a chromosome and the entire chromosome functions as a centromere, called ``holocentromere'' \cite{Plohl2014, Fukagawa2014}.


\subsection*{Centromere specification and role of underlying DNA sequences}
  Understanding the machinery of centromere formation is still ongoing challenge \cite{Plohl2014, Fukagawa2014, McKinley2016}. The chromosomal position where the centromere forms is considered to be specified by the cooperation of centromere-specific histone variant CENP-A, histone modifications and underlying DNA sequences \cite{Fukagawa2014}. However, specific DNA sequences are not considered to be indispensable for the centromere formation, which is inferred by the observations of several ``atypical'' centromeres.

  Neocentromeres are the centromeres that form on ectopic sites distantly from original centromeres along with a loss or an inactivation of the original ones and have been widely observed in many species \cite{Marshall2008, Scott2014}. Neocentromeres can form on loci that do not have repetitive sequences or specific sequence motifs. Dicentric chromosomes have also been widely observed, which possess two regions with centromeric sequences; one of them directs the centromere formation whereas the other one remains inactivated \cite{Earnshaw1985, Steiner1994, Han2006}. It is also known that some species have a few chromosomes with repeat-free centromeres whereas their other chromosomes have repeat-based centromeres \cite{Piras2010, Shang2010, Locke2011}.

  The wide prevalence of repetitive DNA among eukaryote species despite these seeming dispensability of it for centromere formation suggest that repetitive DNA may stabilize the centromere, rather than specify its formation. Some researchers have proposed that the repetitive DNA may serve as a wide range of ``safe'' region with little genes thus provide plasticity for sliding of kinetochore formation \cite{Plohl2014, Fukagawa2014}. It has been hypothesized that tandem repeats occur spontaneously at any genomic position and amplify by unequal crossover between sister chromatids, homologous chromosomes and non-homologous chromosomes \cite{Smith1976, Willard1991, Charlesworth1994}; therefore the naturally-expanded repetitive regions may be selected as a suitable ``safe'' region for the centromere formation. Moreover, the neocentromeres and repeat-free centromeres may be premature states of newly-formed centromeres that will gradually acquire repetitive DNA \cite{Fukagawa2014}. On the other hand, the presence of a widely-conserved centromere protein binding motif (CENP-B box) in centromeric satellites \cite{Henikoff2001} and several recent functional studies \cite{Henikoff2015, Aldrup-MacDonald2016} imply more direct contribution of the underlying DNA sequences to the centromere formation and function.


\subsection*{Early studies}
  Fundamental understanding of centromeric sequence characteristics was established by a number of early studies in 1980's and 1990's, mainly targeting human centromeres. These studies based on experimental methods such as genomic fragmentation by restriction enzymes, pulsed field gel electrophoresis and DNA hybridization.

  The findings include approximate estimation of repeat array size and its divergence among individuals \cite{Oakey1990, Mahtani1990, Greig1991}, the presence of chromosome-specific alpha-satellite HORs \cite{Willard1987} and super-chromosomal subfamilies of alpha-satellites \cite{Alexandrov1988, Alexandrov2001}. It was also revealed that alpha-satellite is widely shared within primate species and even HOR patterns are shared with closely-related species including chimpanzee, gorilla and orangutan \cite{Willard1991}.


\subsection*{The genome projects era}
  Although the basic characteristics of centromeric sequences were revealed by the early studies, understanding detailed sequence organization of the centromere has been challenging in many species, due to the difficulty of assembling its highly-repetitive sequences. In the human genome project, which declared its completion in 2003, large portion of centromeric sequences were missed as huge gaps \cite{M.KatharineRuddand2004, She2004}. Whereas divergent monomeric portion around pericentromeres were assembled in many chromosomes, the assemblies reached more homogeneous HOR regions in a limited number of chromosomes. Nevertheless, subsequent analyses on these few chromosomes revealed sequence landscapes with never-seen resolution \cite{Schueler2001, Ross2005, Nusbaum2006, Rudd2006}. The unsatisfactory sequence assembly at the repetitive centromeric regions were the case in other contemporarily-assembled species \cite {Waterston2002, Hoskins2007}.


\subsection*{Second-generation sequencing-based studies}
  Although second-generation sequencers (SGSs) represented by Illumina and 454 dramatically lowered sequencing cost and accomplished an increased number of genome assemblies \cite{Schatz2010}, they achieved virtually no improvement in centromeric sequence assembly because of their short read length. Nevertheless, their high throughput sequencing combined with chromatin immunoprecipitation (ChIP-seq) facilitated identification of centromere-associated sequences and characterization of functional regions in the assembled centromeric sequences \cite{Hayden2013}.


\subsection*{Computational studies}
  A number of computational studies on centromeric sequences were conducted using Sanger and Illumina whole genome shotgun (WGS) sequencing data, some of which made remarkable achievement. Some studies identified candidate centromeric satellite sequences from WGS data \cite{Alkan2011, Melters2013}, whereas others identified novel HOR patterns from assembled sequences \cite{Rosandic2003} or from WGS data \cite{Alkan2007}.

  Melters \textit{et al}. \cite{Melters2013} identified candidate centromeric satellite sequences of 282 species (204 animals and 78 plants) using WGS data from various sequencing platforms, mainly from Sanger and Illumina. They based on the assumption that the most abundant tandem repeat in a genome derives from centromeric sequences, which is true for the most species whose centromeric sequences has been previously characterized. This study revealed that centromeric satellites from various eukaryotic species do not share common properties such as repeat unit length, GC content or genomic abundance and that centromeric satellite sequences are conserved among only closely-related species of within 50 million years after separation. These results confirmed a traditional view that centromeric sequences evolves rapidly, independently of the other genomic regions \cite{Henikoff2001}.

  Another remarkable computational study is from Miga \textit{et al}. \cite{Miga2014}, in which they generated centromeric array sequences of each human chromosome, using graph-based probabilistic models constructed from Sanger WGS reads. Although the generated sequences do not guarantee long-range ordering of the satellite sequences, they adequately represent local ordering, thus provide useful scaffolds for mapping sequencing reads and other downstream analyses. The generated centromeric sequences have been positioned in the centromeric regions in the latest human reference genome (GRCh38) \cite{GenomeRef2013}.


\subsection*{Long-read sequencing}
  Despite these developments in sequencing technologies and computational methods, long-range organization of centromeric sequences could not be resolved mainly due to the short read length of Sanger or SGS technologies. However, recently-emerging long-read technologies of PacBio \cite{Eid2009} and Oxford Nanopore \cite{Jain2016} are expected as promising tools for advancing centromere studies \cite{Aldrup-MacDonald2014, Miga2015}.

  PacBio single-molecule real-time (SMRT) sequencing yields average read length of $\sim$15kb and longest of $\sim$50kb with P6-C4 chemistry. This long read length enables to capture long-range structure such as HORs directly and provides more opportunity to anchor repetitive sequences to adjacent unique regions. Although error rate of PacBio sequencing is relatively high ($\sim$15\%), the error pattern is believed to be random, thus can be corrected with enough sequencing coverage \cite{Myers2014}. Furthermore, in contrast to Sanger or SGS technologies which suffer from sequencing bias deriving from PCR amplification and/or vector cloning steps, amplification-free PacBio sequencing involves no apparent sequencing bias \cite{Ross2013}. In recent years, a number of studies have reported dramatic improvement in genome assemblies using PacBio sequencing, some of which improved centromere assembly \cite{VanBuren2015, Vij2016, Jiao2016}. Among these, a \textit{de novo} assembly of a grass genome covered three of the nine centromeric regions which are comprised mainly of 155-bp satellite sequences and spans $\sim$400 kb \cite{VanBuren2015}. Some studies targeting centromere-associated repeat-rich regions using PacBio sequencing have also been reported recently \cite{Wolfgruber2016, Khost2016}, and a computational tool for detecting HOR patterns within long reads was also developed \cite{Sevim2016}.

  Oxford Nanopore sequencing yields even longer read length (e.g. MinION sequencer routinely yields $>$150 kb read) with accuracy of $\sim$92\% \cite{Jain2016}. However, some early-adopting laboratories of it including the author's laboratory observe unstable read length and much higher error rate than the officially announced rate (Kin Fai Au, personal communication), thus the community awaits improvement and sophistication of its sequencing technology and protocols.


\subsection*{Limited number of in-depth studies, especially in fish}
  Due to the long-standing difficulty in assembling centromeric regions, in-depth analysis of centromeric sequences have been limited to a relatively-small number of species, including human, some other primates, mouse, some plants, \textit{Drosophila} and yeasts\cite{Plohl2014}. In fish species, detailed study on centromeric sequence organization has been scarce. Centromere-associated satellites have been identified in zebrafish, seabass and stickleback by CENP-A-targeted ChIP and/or FISH experiments \cite{Phillips2000, Kuznetsova2014, Cech2015}, however only limited amount of them have been captured in assembled genomes \cite{Howe2013, Vij2016, Cech2015}.

  Medaka, as well as zebrafish, has traditionally played an important role as a model organism of fish species \cite{Wittbrodt2002}. Cloning-based assembly of its genome was conducted about ten years ago \cite{Kasahara2007}, however centromeric regions were largely missed in gaps, as was the case in many contemporary assembly projects. A 156-bp candidate centromeric satellite sequence of medaka was identified in the computational study by Melters \textit{et al} \cite{Melters2013}, though whether this sequence truly derives from the centromere has not been confirmed by experimental methods such as ChIP-seq or FISH.


\subsection*{This study}
  The author's laboratory recently assembled three medaka inbred strain genomes using PacBio long reads and achieved dramatic improvement in the assembly quality (Ichikawa \textit{et al.}, unpublished). Based on these high quality medaka genomes, this study conducted in-depth analysis of medaka centromeric sequences. This study revealed the presence of inter-chromosomal relationship of the satellite sequences and its conservation among the strains. Also the evidence of higher-order structure (HORs) was observed.
